%\documentclass[a4paper, 11pt, addpoints]{exam}
%%\documentclass[a4paper, 11pt, addpoints, answers]{exam}  % Desconmente esta linha, para ver as respostas, e comente a de cima
%\usepackage{listaufersa}
%\usepackage{listings} % Mostrar código-fonte
%\usepackage[brazil]{babel}
%\usepackage{multicol}
%\usepackage{booktabs}
%\usepackage{hyperref}
%
%\setlength{\columnsep}{25pt}
%\lstdefinestyle{js}{
%    basicstyle=\ttfamily,
%    breaklines=true,
%    breakatwhitespace=true,
%    tabsize=1,
%    resetmargins=true,
%    xleftmargin=0pt,
%    frame=none
%}
%
%
%
%
%%%%%%%%%%%%%%%%%%%%%%%%%%%%%%%%%%%%%%%%%%%%%%%%%%
%
%\pointpoints{Ponto}{Pontos}
%
%
%
%
%
%\begin{document}
%
%
%
%%%%%%%%%% Informações sobre o curso %%%%%%%%%%%%%
%
%
%\nomeProfessor{Rodrigo Toledo Teixeira Câmara}
%\nomeCurso{Bacharelado em Ciências e Tecnologias}
%\nomeDisciplina{Cálculo Numérico}
%\semestre{5}   % Deixe em branco se for para mais de um semestre (recuperação)
%\dataDaProva{2016.2}
%\tipoAvaliacao{lista da aula 14}
%
%\info
%\vspace{-0.5 cm}
%
%
%%%%%%%%%%%%%%%%%%%%%%%%%%%%%%%%%%%%%%%%%%%%%%%%%%
%
%
%
%\begin{questions}
%%\begin{multicols}{2}

\begin{ex}
Calcule uma aproximação do número $I$, dado por $$I=\int_0^4 x^2 dx,$$ utilizando a regra de 1/3 de Simpson. Em seguida, estime o erro. Por fim, calcule $I$ utilizando o Teorema Fundamental do Cálculo para confirmar se a previsão do erro está razoável. Compare os resultados com os obtidos com a integração pela regra do trapézio (questão 1 da lista 13).
\end{ex}

\begin{ex}
Utilizando a regra de 1/3 de Simpson repetida, calcule o valor aproximado do número $\int_0^6 (\cos(x) + x)dx$, usando $6$ pontos. Em seguida, estime o erro desta aproximação. Compare com os resultados obtidos com a integração pela regra dos trapézios repetidos (questão 2 da lista 13).
\end{ex}

\begin{ex}
Considere a função $f$ calculada em alguns pontos, descritos pela tabela \ref{inte}.

% Please add the following required packages to your document preamble:
% \usepackage{booktabs}
\begin{table}[htb]
\centering
\caption{Alguns pontos da função $f$.}
\label{inte}
\begin{tabular}{@{}llllllll@{}}
\toprule
$x$    & 0    & 1    & 2    & 3    & 4    & 5    & 6    \\ \midrule
$f(x)$ & 0.21 & 0.32 & 0.42 & 0.51 & 0.82 & 0.91 & 1.12 \\ \bottomrule
\end{tabular}
\end{table}

Calcule $\int_0^6 f(x)dx$ pela regra de 1/3 de Simpson repetida.
\end{ex}


\begin{ex}
É possível estime $\int_0^1 3e^{-x}dx$ utilizando a regra de 1/3 de Simpson repetida com $h=1/5$? Se for, calcule. Caso contrário, sugira outro valor de $h$.
\end{ex}


\begin{ex}
Em quantos subintervalos, no mínimo, precisamos dividir o domínio de integração $[0,2]$ para que o erro da integração pela regra de 1/3 de Simpson repetida de $\int_0^{2}e^x$ seja menor que $0.01$?
\end{ex}

\begin{ex}
Considere a função $f$ dada pela tabela \ref{diferente}.

\begin{table}[htb]
\centering
\caption{Alguns pontos da função $f$.}
\label{diferente}
\begin{tabular}{@{}lllll@{}}
\toprule
$x$    & 0.1    & 0.2    	& 0.33    & 0.5    \\ \midrule
$f(x)$ & -2.3 	& -1.6		& -1.1 & -0.4  \\ \bottomrule
\end{tabular}
\end{table}
\end{ex}
Calcule $\int_{0.1}^{0.5}f(x)dx$ (atenção: observe que os pontos do domínio na tabela \emph{não estão} igualmente espaçados).
%\end{questions}
%
%\end{document}