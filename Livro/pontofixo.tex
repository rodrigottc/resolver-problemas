%\documentclass[a4paper, 11pt, addpoints]{exam}
%%\documentclass[a4paper, 11pt, addpoints, answers]{exam}  % Desconmente esta linha, para ver as respostas, e comente a de cima
%\usepackage{listaufersa}
%\usepackage{listings} % Mostrar código-fonte
%\usepackage[brazil]{babel}
%\usepackage{multicol}
%\usepackage{booktabs}
%\usepackage{hyperref}
%
%\setlength{\columnsep}{25pt}
%\lstdefinestyle{js}{
%    basicstyle=\ttfamily,
%    breaklines=true,
%    breakatwhitespace=true,
%    tabsize=1,
%    resetmargins=true,
%    xleftmargin=0pt,
%    frame=none
%}
%
%
%
%
%%%%%%%%%%%%%%%%%%%%%%%%%%%%%%%%%%%%%%%%%%%%%%%%%%
%
%\pointpoints{Ponto}{Pontos}
%
%\begin{document}
%
%
%%%%%%%%%% Informações sobre o curso %%%%%%%%%%%%%
%
%
%\nomeProfessor{Rodrigo Toledo Teixeira Câmara}
%\nomeCurso{Bacharelado em Ciências e Tecnologias}
%\nomeDisciplina{Cálculo Numérico}
%\semestre{5}   % Deixe em branco se for para mais de um semestre (recuperação)
%\dataDaProva{2016.2}
%\tipoAvaliacao{lista da aula 07}
%
%\info
%\vspace{-0.5 cm}
%
%
%%%%%%%%%%%%%%%%%%%%%%%%%%%%%%%%%%%%%%%%%%%%%%%%%%
%
%
%
%\begin{questions}
%%\begin{multicols}{2}


\begin{ex}
Considere a função dada $f$ dada por $$f(x)=x^6+4x^2-20.$$
\begin{enumerate}
\item Trace o gráfico desta função no \emph{GeoGebra}  e estime a menor raiz positiva. 
\item Utilizando a forma geral $$\varphi(x)=x+A(x)f(x),$$ com $A(\xi)\neq 0$, construa três funções iterativas $\varphi$.
\item Verifique se alguma destas funções iterativas converge para a menor raiz positiva de $f$.
\item Determine a função $\varphi$ conforme proposta pelo método de Newton-Raphson.
\item Calcule as duas primeiras iterações deste método.
\end{enumerate}
\end{ex}






\begin{ex}
Encontre a raiz quadrada de 7 com treze algarismos significativos. Dica: a raiz quadrada de $a$ é encontrada ao se calcular a raiz da função $f(x)=x^2-a$.
\end{ex}

\begin{ex}
Encontre a raiz cúbica de 10 com treze algarismos significativos. Dica: a raiz $n$-ésima de $a$ é encontrada ao se calcular a raiz da função $f(x)=x^n-a$.
\end{ex}

%\begin{ex}
%A velocidade de um paraquedista, após abrir o paraquedas, é dada pela equação
%$$v(t)=\frac{gm(1-e^{-(\frac{ct}{m})})}{c},$$
%onde \begin{itemize}
%\item $v$ é a velocidade absoluta em $m/s$ do paraquedista, 
%\item $g$ é a aceleração da gravidade, 
%\item $m$ é a massa do paraquedista (já com o equipamento), 
%\item $t$ é o tempo em segundos após a abertura do paraquedas e 
%\item $c$ é o coeficiente de amortecimento, em $kg/s$, que depende, principalmente, do paraquedas.
%\end{itemize}
%Considerando a aceleração da gravidade como $9.8 m/s^2$, a massa do paraquedista com o equipamento é 100$kg$ e que ele atinge a velocidade de $40 m/s$ sete segundos após abrir o paraquedas, encontre o coeficiente de amortecimento deste paraquedas.
%\end{ex}

\begin{ex}
A equação que relaciona a taxa de juros $i$ com o valor à vista $PV$, o valor fixo\footnote{Este método de pagamento, que envolve prestações iguais, é conhecido como o \emph{sistema Price}.} $PMT$ de cada parcela e a quantidade de parcelas $n$ é
$$PMT=PV\frac{i(1+i)^n}{(1+i)^n-1}.$$
Supondo que uma geladeira é vendida à vista por R$\$1500,00$ e que a loja permite o pagamento em doze parcelas iguais de R$\$150,00$, calcule a taxa de juros deste financiamento. 
\end{ex}

\begin{ex}
Um espectrômetro analisou as ondas de uma certa transmissão de rádio e as descreveu como a seguinte função senoidal:
$$y(x)=x^3\sin(x)+\cos(x)+3.$$
Trace esta função no \emph{GeoGebra} para observar uma representação visual destas ondas da rádio. 


O \emph{comprimento de onda}, $\lambda$, em metros, é definido, neste contexto, como o dobro da distância horizontal entre duas raízes consecutivas desta função. A \emph{frequência} $f$, em hertz, é dada por
$$f=\frac{c}{\lambda},$$
onde $c$ é a velocidade da luz, dada por $c=3\times 10^8 m/s$.


Determine a frequência desta transmissão de rádio.
\end{ex}
%
%\end{questions}
%
%\end{document}