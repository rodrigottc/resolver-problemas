%\documentclass[a4paper, 11pt, addpoints]{exam}
%%\documentclass[a4paper, 11pt, addpoints, answers]{exam}  % Desconmente esta linha, para ver as respostas, e comente a de cima
%\usepackage{listaufersa}
%\usepackage{listings} % Mostrar código-fonte
%\usepackage[brazil]{babel}
%\usepackage{multicol}
%\usepackage{booktabs}
%\usepackage{hyperref}
%
%\setlength{\columnsep}{25pt}
%\lstdefinestyle{js}{
%    basicstyle=\ttfamily,
%    breaklines=true,
%    breakatwhitespace=true,
%    tabsize=1,
%    resetmargins=true,
%    xleftmargin=0pt,
%    frame=none
%}
%
%
%
%
%%%%%%%%%%%%%%%%%%%%%%%%%%%%%%%%%%%%%%%%%%%%%%%%%%
%
%\pointpoints{Ponto}{Pontos}
%
%
%
%
%
%\begin{document}
%
%
%
%%%%%%%%%% Informações sobre o curso %%%%%%%%%%%%%
%
%
%\nomeProfessor{Rodrigo Toledo Teixeira Câmara}
%\nomeCurso{Bacharelado em Ciências e Tecnologias}
%\nomeDisciplina{Cálculo Numérico}
%\semestre{5}   % Deixe em branco se for para mais de um semestre (recuperação)
%\dataDaProva{2016.2}
%\tipoAvaliacao{lista da aula 12}
%
%\info
%\vspace{-0.5 cm}
%
%
%%%%%%%%%%%%%%%%%%%%%%%%%%%%%%%%%%%%%%%%%%%%%%%%%%
%
%
%
%\begin{questions}
%%\begin{multicols}{2}
%



%\begin{ex}
%Um grupo de alunos de biologia estudou a bioquímica de uma certa espécie de alga, a \emph{Chlorophyta Caulerpa cupressoides}. Eles observaram que a salinidade (em $PSU$) da água influencia nas atividades biológicas da alga, em especial na síntese de uma substância chamada \emph{polissacarimídia} (em milimol por hora). Os resultados das experiências estão descritos na tabela \ref{alga}.
%
%% Please add the following required packages to your document preamble:
%% \usepackage{booktabs}
%\begin{table}[htb]
%\centering
%\caption{Experiências que relacionam a capacidade da alga \emph{Chlorophyta Caulerpa cupressoides} sintetizar polissacarimídia em função da salinidade.}
%\label{alga}
%\begin{tabular}{@{}ll@{}}
%\toprule
%Salinidade & Síntese de polissacarimídia \\ \midrule
%3.2        & 0.9                         \\
%0.6        & 3.9                         \\
%1.3        & 2.8                         \\
%2.3        & 2.1                         \\
%3.1        & 1.6                         \\ \bottomrule
%\end{tabular}
%\end{table}
%Ao ajustar estes dados por uma curva hiperbólica, obtém-se $\varphi(x)=1/(0.3x+0.1)$. Ao se ajustar por uma reta, encontra-se $\varphi(x)=-x+4.3$. Segundo o critério dos mínimos dos desvios quadrados, qual dessas duas curvas se ajusta melhor a este fenômeno biológico? 
%\end{ex}


\begin{ex}
A tabela \ref{condutividade} relaciona a condutividade elétrica específica de um certo material em função da temperatura. Sabendo que a equação $$\lambda=b^{t}a$$ relaciona essas duas grandezas, determine as constantes $b$ e $a$ deste material.

% Please add the following required packages to your document preamble:
% \usepackage{booktabs}
\begin{table}[htb]
\centering
\caption{Condutividade elétrica $\lambda$ em função da temperatura (em graus Celsius).}
\label{condutividade}
\begin{tabular}{@{}llllllll@{}}
\toprule
$t$       & 14.5 & 30.0  & 64.5  & 74.5  & 86.7  & 94.5  & 98.9  \\ \midrule
$\lambda$ & 0    & 0.004 & 0.018 & 0.029 & 0.051 & 0.073 & 0.090 \\ \bottomrule
\end{tabular}
\end{table}



\end{ex}


\begin{ex}
Ao ser dada a largada de uma corrida de Fórmula 1, os sensores instalados no carro de uma determinada equipe emitiram os dados descritos na tabela \ref{f1}:

% Please add the following required packages to your document preamble:
% \usepackage{booktabs}
\begin{table}[htb]
\centering
\caption{A relação consumo/velocidade do primeiro segundo de uma corrida de Fórmula 1.}
\label{f1}
\begin{tabular}{@{}llr@{}}
\toprule
Tempo ($s$) & Consumo ($l/s$) & Velocidade ($km/h$) \\ \midrule
0.1         & 0.015           & 5.000               \\
0.2         & 0.019           & 5.727               \\
0.3         & 0.024           & 6.681               \\
0.4         & 0.031           & 7.979               \\
0.5         & 0.041           & 9.831               \\
0.6         & 0.057           & 12.643              \\
0.7         & 0.083           & 17.292              \\
0.8         & 0.133           & 25.988              \\
0.9         & 0.247           & 45.633              \\
1.0         & 0.599           & 106.200             \\ \bottomrule
\end{tabular}
\end{table}
Faça hipóteses para a modelagem do consumo em função do tempo e para a modelagem da velocidade em função do tempo por meio do método dos mínimos quadrados.
 
\end{ex}
\begin{ex}
Em 1965, Gordon Moore, um dos fundadores da \emph{Intel}, propôs uma previsão para a crescente miniaturização dos \emph{chips} que ficou conhecida como a ``Lei de Moore''. Em sua ``profecia'', Moore afirmava que o número de transistores dos \emph{chips} teriam o número duplicado a cada 18 meses. A tabela \ref{chips} relaciona alguns \emph{chips} com o ano de lançamento.


De posse destes dados, proponha uma função que estabeleça uma previsão da capacidade dos \emph{chips} de 2020. (Dica: procure uma função $N=f(a)$, onde $N$ é o número de transistores e $a$ é o ano. Antes de tudo, transforme o eixo de transistores para $log_{10}(N)$).
% Please add the following required packages to your document preamble:
% \usepackage{booktabs}
\begin{table}[]
\centering
\caption{Relação entre ano de lançamento de \emph{chips} e o número de transistores.}
\label{chips}
\begin{tabular}{@{}llr@{}}
\toprule
\emph{Chip}        & Ano  & Número de transistores \\ \midrule
4004                 & 1971 & 2 250                  \\
8008                 & 1972 & 3 300                  \\
8080                 & 1974 & 6 000                  \\
8086                 & 1978 & 29 000                 \\
80286                & 1982 & 134 000                \\
80386                & 1986 & 275 000                \\
80486                & 1989 & 1 200 000              \\
\emph{Pentium}     & 1993 & 3 100 000              \\
\emph{Pentium II}  & 1997 & 7 500 000              \\
\emph{Pemtium III} & 1999 & 9 500 000              \\
\emph{Pentium 4}   & 2000 & 42 000 000             \\ \bottomrule
\end{tabular}


Fonte: BURIAN, Reinaldo; DE LIMA, Antonio Carlos; JÚNIOR, Annibal Hetem. \emph{Cálculo numérico}. Livros Técnicos e Científicos, 2007.
\end{table}

\end{ex}

%
%\end{questions}
%
%\end{document}