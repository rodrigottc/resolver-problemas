\documentclass[10pt,a4paper]{report}
\usepackage[utf8]{inputenc}
\usepackage[portuguese]{babel}
\usepackage[T1]{fontenc}
\usepackage{amsmath}
\usepackage{amsfonts}
\usepackage{amssymb}
\usepackage{graphicx}
\usepackage{lmodern}
\author{Rodrigo Câmara}

\usepackage{amsthm}
\usepackage{multicol}
\usepackage{color}
\usepackage{booktabs}

\usepackage[pdftex]{hyperref}
\usepackage{import}
\usepackage{enumerate}
\usepackage{enumitem}
\usepackage{answers}
\graphicspath{{fig/}}


\Newassociation{sol}{Solucao}{ans}
\theoremstyle{definition}
\newtheorem{ex}{Q}[section]
\title{Exercícios e problemas de Cálculo Numérico}
\begin{document}
\frenchspacing

\maketitle
\tableofcontents
\Opensolutionfile{ans}[ans1]
\chapter*{Introdução}
%\addcontentsline{toc}{chapter}{Introdução}

Este livro traz uma coleção de exercícios e problemas sobre Cálculo Numérico, desenvolvidos para alunos de graduação. As questões foram ou elaboradas por mim ou adaptadas de livros didáticos\footnote{ARENALES, Selma; DAREZZO, Artur. \emph{Cálculo numérico: aprendizagem com apoio de software}. Cengage Learning, 2008.\\ RUGGIERO, Márcia A. Gomes; LOPES, Vera Lúcia da Rocha. \emph{Cálculo numérico: aspectos teóricos e computacionais}. Makron Books do Brasil, 1997.\\ BURIAN, Reinaldo; DE LIMA, Antonio Carlos; JÚNIOR, Annibal Hetem. \emph{Cálculo numérico}. Livros Técnicos e Científicos, 2007.}. Alguns exercícios (não todos) possuem a solução nas páginas finais. Observe que é \emph{apenas} a solução. Cabe ao aluno escrever o encadeamento de argumentos que levam à ela.

Este material é \emph{grátis} e \emph{livre}, sob a licença \emph{GNU General Public License v3.0}. Isso significa que você pode utilizar e modificar este trabalho à vontade, desde que dê os devidos créditos. Eu também gostaria de receber críticas, sugestões e relatos de sua experiência utilizando este livro, seja você aluno ou professor. Entre em contato comigo pelo email \url{rodrigo.camara@ufersa.edu.br}. O código fonte deste livro, escrito em linguagem \LaTeX, está disponível em \url{https://github.com/rodrigottc/resolver-problemas}.

Preciso avisar que este livro está em versão \emph{beta}, ou seja, ainda está \emph{incompleto} e possivelmente com erros. Por favor, caso encontre algum, avise-me. Verifique sempre a data que consta na primeira página.


Este livro foi feito por dois motivos: o primeiro, e mais importante, é oferecer uma forma fácil para que os alunos de graduação do curso \emph{Cálculo Numérico} tenham acesso aos exercícios. Não canso de enfatizar a importância de exercitar o que foi aprendido em aula. Como um professor uma vez me ensinou, ``aprender Matemática é como aprender a nadar: não basta apenas ver''. 


O segundo motivo é puramente por organização própria. Percebi que se não encontrasse uma forma de reunir as questões que elaborei, elas possivelmente se perderiam no meio de códigos e arquivos duplicados no meu computador.


%Você notará que algumas questões são parecidas, com apenas mudanças em alguns valores. Isso é proposital. Resolvi seguir esta abordagem por perceber que, sem repetição de questões parecidas, os alunos enxergam padrões onde não deveria existir.
\subsection*{Como usar este livro}
As questões são divididas em dois tipos: \emph{exercícios} e \emph{problemas}. As do tipo \emph{Exercícios} trazem questões simples e têm a finalidade de treinar o aluno para a teoria nova. As do tipo \emph{Problemas} são as mais interessantes. Cada problema vai exigir criatividade para relacionar o que ele aprendeu nos exercícios com o que está sendo pedido.


A resolução de problemas matemáticos é de preponderante importância para a educação, pois oferece suporte à curiosidade dos estudantes, ao mesmo tempo em que traz situações reais para a sala de aula, propicia a possibilidade da descoberta do novo e incentiva o gosto pelo desafio mental. Segundo George Polya, matemático húngaro considerado percurssor desta didática, 

\begin{quotation}
``Uma grande descoberta resolve um grande problema, mas há sempre uma pitada de descoberta na resolução de qualquer problema. O problema pode ser modesto, mas se ele desafiar a curiosidade e puser em jogo as faculdades inventivas, quem o resolver por seus próprios meios, experimentará a tensão e gozará o triunfo da descoberta. Experiências tais, numa idade susceptível, poderão gerar o gosto pelo trabalho mental e deixar, por toda a vida, a sua marca na mente e no caráter\footnote{POLYA, G. A arte de resolver problemas. 2a reimpressão, 2a . ed. Rio de Janeiro:
Interciência, 1995.}''.
\end{quotation}
Recomendo a leitura da obra de Polya, mas, resumidamente, ele propõe a seguinte estratégia para resolver problemas:
\begin{quote}
\subsubsection{1. Entenda o problema}
        Primeiro, você tem de entender o problema:
\begin{itemize}
\item        Qual é a incógnita? Quais são os dados? Quais são as condições? Mais importante: descubra o que está sendo pedido!
\item        É possível responder o que está sendo pedido? As condições são suficientes para determinar a incógnita? Ou são insuficientes? Ou redundantes? Ou contraditórias?
\item        Faça uma figura. Outra se necessário. Introduza notação adequada.
\item        Separe as condições em partes
\end{itemize}

\subsubsection{2. Construa uma estratégia de resolução}    
Ache conexões entre os dados e a incógnita. Talvez seja conveniente considerar problemas auxiliares ou particulares, se uma conexão não for achada em tempo razoável. Use isso para ``bolar'' um plano ou estratégia de resolução do problema.
\begin{itemize}

 \item        Você já encontrou este problema ou algum parecido?
  \item      Você conhece um problema semelhante? Conhece teoremas ou fórmulas que possam ajudar?
 \item        Olhe para a incógnita! Tente achar um problema familiar e que tenha uma incógnita semelhante
 \item       Digamos que você encontrou um problema, já resolvido, relacionado com o seu. É possível aproveitá-lo? Reutilizar o método? Deve-se introduzir algum elemento auxiliar de modo a viabilizar esses objetivos?
  \item      Você consegue enunciar o problema de uma outra maneira?
 \item       Se não consegue resolver o problema dado, tente resolver um problema parecido. Você consegue imaginar um caso particular mais acessível? Um caso mais geral e mais acessível? Consegue resolver alguma parte do problema? Mantenha apenas parte das condições do problema e observe o que ocorre com a incógnita, como ela varia agora? Você consegue obter alguma coisa desde os dados? Consegue imaginar outros dados capazes de produzir a incóognita? Consegue alterar a incógnita ou os dados, ou ambos, de modo que a nova incógnita e os novos dados fiquem mais próximos?
 \item       Você está levando em conta todos os dados? E todas as condições?
\end{itemize}


    \subsubsection{3. Execute a estratégia}

    Frequentemente, esta é a etapa mais fácil do processo de resolução de um problema. Contudo, a maioria dos principiantes tendem a pular para essa etapa prematuramente, e acabam dando-se mal. Outros elaboram estratégias inadequadas e acabam se enredando terrivelmente na execução.

\begin{itemize}
 \item      Execute a estratégia.
  \item      Ao executar a estratégia, verifique cada passo. Você consegue mostrar claramente que cada um deles está correto?

\end{itemize}

    \subsubsection{4. Revise}
Examine a solução obtida.
\begin{itemize}
\item        Verifique o resultado e o argumento
\item        Você pode obter a solução de um outro modo?
\item        Qual a essência do problema e do método de resolução empregado? Em particular, Você consegue usar o resultado, ou o método, em algum outro problema?
\end{itemize}
\end{quote}
Espero que este material que você tem em mãos o auxilie neste curso. Qualquer dúvida, não hesite em conversar comigo. Estou na sala 11 do bloco 1 e respondo os \emph{emails} enviados para \url{rodrigo.camara@ufersa.edu.br} o mais rápido possível.

\begin{flushright}
-- Rodrigo Câmara.
\end{flushright}
\chapter*{Histórico de mudanças}
\begin{description}
\item[26/06/17] \begin{itemize}
\item Primeira versão publicada. O capítulo \ref{ch.erros} sobre erros está completamente utilizável. 
\end{itemize}
\item[05/07/17] Os capítulos \ref{ch.sl} e \ref{ch.funcao} estão completos. Com isso, toda Parte I está utilizável.
\begin{itemize}
\item Mudei algumas funções da questão \ref{primeirasitbi}. Algumas delas não tinham raízes.
\item Novo problema: Q \ref{prob.coluna}. Esse problema foi usado pela primeira vez em 2016.1. Esta versão foi simplificada.
\item Novo problema: Q \ref{prob.matfin}, sobre matemática financeira.
\item Novo problema: Q \ref{prob.vigas}, um elegante (e desafiador) problema escrito por RUGGIERO.
\item Novo problema: Q \ref{ex.quimica}, sobre balanço de reações químicas.
\item Novo problema: Q \ref{ex.forno}, um problema baseado na monografia do aluno Pedro Henrique, da UFERSA Angicos.
\item Troquei a ordem dos capítulos sobre sistemas lineares e sobre raízes de funções, para seguir a ordem apresentada em ARENALES.
\item Mais soluções para exercícios sobre resolução de sistemas lineares por métodos diretos.
\item Correção do gabarito da questão \ref{ex.normalizada}.
\end{itemize}
\end{description}
\part{Primeira unidade}
\chapter{Noções básicas sobre erros}\label{ch.erros}
\import{./}{aritmetica.tex}
\chapter{Métodos para encontrar solução de sistema linear}\label{ch.sl}
	\section{Métodos diretos}
\import{./}{metodosdiretos.tex}
	\section{Métodos iterativos}
\import{./}{iterativos.tex}
\chapter{Métodos para encontrar raiz de função}\label{ch.funcao}
\import{./}{bisseccao.tex}


%\import{./}{pontofixo.tex}
\part{Segunda unidade}
\chapter{Aproximação de funções}

	\section{Interpolação de funções}
	\import{./}{interpolacao.tex}
	\section{Ajuste de curvas pelo critério dos mínimos quadrados}
	\import{./}{minimosquadrados.tex}
	\import{./}{interpolacao_e_minqua.tex}
	\import{./}{minimosquadrados2.tex}
	\import{./}{aproxteoria.tex}
\part{Terceira unidade}
\chapter{Integração numérica}
	\section{Regra do trapézio}
	\import{./}{trapezios.tex}
	\section{Regra de 1/3 de Simpson}
	\import{./}{simpson.tex}
\chapter{Resolução numérica de equações diferenciais ordinárias}
	\section{Métodos de Euler}
	\import{./}{edeuler.tex}	
	\section{Métodos de Runge-Kutta}
	\import{./}{rungekutta.tex}




\Closesolutionfile{ans}
\appendix
\chapter{Soluções}
%\begin{multicols}{2}
\begin{Solucao}{1.1.1}
\begin{enumerate}[label=\alph*)]
\item $(11001)_2$
\item $(1011)_2$
\item $(1111)_2$
\item $(0.0011)_2$
\item $(11001.11)_2$
\item $(0.10011001100110011...)_2.$ A única forma de representar este número exatamente é utilizando \emph{infinitos} algarismos. Em um computador com um número \emph{finito} de algarismos, esse número não pode ser exatamente representado.
\end{enumerate}
\end{Solucao}
\begin{Solucao}{1.1.2}
\begin{enumerate}[label=\alph*)]
\item $(13)_{10}$
\item $(7)_{10}$
\item $(2)_{10}$
\item $(0.375)_{10}$
\item $(7.375)_{10}$
\end{enumerate}
\end{Solucao}
\begin{Solucao}{1.1.3}
\begin{enumerate}[label=\alph*)]
\item $0.1\times 10^3.$
\item $0.158\times 10^{-1}.$
\item $0.101\times 2^3.$
\end{enumerate}
\end{Solucao}
\begin{Solucao}{1.1.4}
\begin{enumerate}[label=\alph*)]
\item $0.12314*10^2$ ou $0.12314E+2$.
\item $0.25000*10^{12}$ ou $0.25000E+12$.
\item Na notação usual, $0.00019$. Na notação de ponto flutuante normalizado, $0.19*10^3$.
\item $1000000$, ou $0.1*10^7$.
\end{enumerate}
\end{Solucao}
\begin{Solucao}{1.1.5}
\begin{enumerate}
[label=\alph*)]
\item $0.1*10^{-1}$.
\item $0.9999*10^2$.
%\item \emph{Dica}: Não se esqueça que o primeiro dígito após o ponto deve ser diferente de zero. Não se esqueça também de contar os números negativos e o zero.
\item Região de \emph{overflow}: Todos os números maiores que $0.999*10^2$ e todos os números menores que $-0.999*10^2$. Região de \emph{underflow}: todos os números maiores que $-0.1*10^{-1}$ e menores que $0.1*10^{-1}$ (exceto o número zero).
\end{enumerate}
\end{Solucao}
\begin{Solucao}{1.1.6}
		\begin{align*}
		&(((4+4)+4)+4)+56070\\
		&=(((0.4*10^1+0.4*10^1)+4)+4)+56070\\
		&=((0.8*10^1+0.4*10^1)+4)+56070\\
		&=(0.12*10^2+0.4*10^1)+56070\\
		&=(0.12*10^2+0.04*10^2)+56070\\
		&=0.16*10^2+0.5607*10^5\\
		&=0.00016*10^5+0.5607*10^5\\
		&=0.56086*10^5\\
		&=0.5609*10^5.
		\end{align*}
		Agora, para os cálculos de $(((56070+4)+4)+4)+4$, observe que $56070+4$ é calculado como
		\begin{align*}
		 &0.5607*10^5+0.4*10^1\\
		 &=0.5607*10^5+0.00004*10^5\\
		 &=0.56074*10^5\\
		 &=0.5607*10^5.
		\end{align*}Chegue à conclusão que a primeira expressão apresentada possui resultado diferente da segunda.
	
\end{Solucao}
\begin{Solucao}{2.1.1}
Pode sim ser escrita como sistema triangular. No caso,
$$\begin{bmatrix}
1&-1&3&8\\
0&5&1&1\\
0&0&4&-2\\
0&0&0&10
\end{bmatrix}.$$
A solução é $X=\begin{bmatrix}
-46&2&4&3\end{bmatrix}^t$.
\end{Solucao}
\begin{Solucao}{2.1.2}
\begin{enumerate}[label=\alph*)]
\item Primeiro reescrevo o sistema linear na forma da matriz aumentada,
$$\begin{bmatrix}
{\color{red}3}	& 2 & 3 & 9 \\
-1			& 4 & 6	& 32 \\
10			& 2 & 12 & -2
\end{bmatrix}.$$
O pivô é o elemento $a_{11}$, pintado de vermelho. Vou usar este pivô para ``zerar'' todos os elementos abaixo dele, por meio das \emph{operações fundamentais}.

O primeiro passo é realizar as operações fundamentais $$L_2\leftarrow L_2 - \frac{-1}{{\color{red}3}}L_1$$ e $$L_3\leftarrow L_3 - \frac{10}{{\color{red}3}}L_1.$$ A matriz fica
$$\begin{bmatrix}
3			& 2 & 3 & 9 \\
0			& {\color{red}14/3} & 7	& 35 \\
0			& -14/3 & 2 & -32
\end{bmatrix}.$$


O segundo passo é usar o pivô $14/3$ para zerar todos os elementos abaixo dele. A operação fundamental é $$L_3\leftarrow L_3 - \frac{-14/3}{{\color{red}14/3}}L_1.$$
A matriz fica
$$\begin{bmatrix}
3			& 2 & 3 & 9 \\
0			& 14/3 & 7	& 35 \\
0			& 0 & 9 & 3
\end{bmatrix},$$
que é uma matriz triangular, fácil de resolver.

\item A matriz $A$, dada por $$A=\begin{bmatrix}
3	&2	&3\\
-1	&4	&6\\
10	&2	&12
\end{bmatrix},$$
pode ser decomposta de maneira única como produto de uma matriz inferior $L$ e uma matriz superior $U$ desta forma:
$$\begin{bmatrix}
3	&2	&3\\
-1	&4	&6\\
10	&2	&12
\end{bmatrix}=
\begin{bmatrix}
{\textcolor{red}1}	&{\textcolor{red}0}	&{\textcolor{red}0}\\
{\textcolor{blue}-1/3 }	&{\textcolor{red}1}	&{\textcolor{red}0}\\
{\textcolor{blue} 10/3 }	&{\textcolor{blue}-1}	&{\textcolor{red}1}
\end{bmatrix}\times
\begin{bmatrix}
3	&2	&3\\
0	&14/3	&7\\
0	&0	&9
\end{bmatrix}.$$
Para construir a matriz $L$, eu tomei os {\textcolor{blue} multiplicadores} que foram utilizados no algoritmo de eliminação de Gauss. A matriz $U$ é simplesmente o resultado do algoritmo de Gauss.


O problema original pode ser reescrito como
$$\begin{bmatrix}
{\textcolor{red}1}	&{\textcolor{red}0}	&{\textcolor{red}0}\\
{\textcolor{blue}-1/3 }	&{\textcolor{red}1}	&{\textcolor{red}0}\\
{\textcolor{blue} 10/3 }	&{\textcolor{blue}-1}	&{\textcolor{red}1}
\end{bmatrix}\times
\begin{bmatrix}
3	&2	&3\\
0	&14/3	&7\\
0	&0	&9
\end{bmatrix}\times
\begin{bmatrix}
x_1\\x_2\\x_3
\end{bmatrix}=
\begin{bmatrix}
9\\32\\-2
\end{bmatrix}.$$
Agora precisamos apenas resolver os dois sistemas lineares \emph{triangulares}: o sistema $Ly=b$,
$$\begin{bmatrix}
{\textcolor{red}1}	&{\textcolor{red}0}	&{\textcolor{red}0}\\
{\textcolor{blue}-1/3 }	&{\textcolor{red}1}	&{\textcolor{red}0}\\
{\textcolor{blue} 10/3 }	&{\textcolor{blue}-1}	&{\textcolor{red}1}
\end{bmatrix}\times
\begin{bmatrix}
a\\b\\c
\end{bmatrix}=
\begin{bmatrix}
9\\32\\-2
\end{bmatrix},$$
cuja solução é $a=9, b=35, c=3$, e este outro, $Ux=y$,
$$\begin{bmatrix}
3	&2	&3\\
0	&14/3	&7\\
0	&0	&9
\end{bmatrix}\times
\begin{bmatrix}
x_1\\x_2\\x_3
\end{bmatrix}=
\begin{bmatrix}
9\\35\\3
\end{bmatrix}
$$
cuja solução (que é a solução que procuramos) é $x_1=-2, x_2=7, x_3=1/3$.
\end{enumerate}
\end{Solucao}
\begin{Solucao}{2.1.3}
A matriz $A$ dos coeficientes, dada por $\begin{bmatrix}
2&3\\4&5
\end{bmatrix}$, pode ser decomposta como
$$\begin{bmatrix}
2&3\\4&5
\end{bmatrix}=\begin{bmatrix}
1&0\\2&1
\end{bmatrix}\begin{bmatrix}
2&3\\0&-1
\end{bmatrix}.$$

A solução do primeiro sistema é $X=\begin{bmatrix}
1&1
\end{bmatrix}^t$. A solução do segundo sistema é $X=\begin{bmatrix}
-1&6
\end{bmatrix}^t$.
\end{Solucao}
\begin{Solucao}{2.1.4}
$-4$.
\end{Solucao}
\begin{Solucao}{2.2.1}
\begin{enumerate}
\item A análise pelo critério das linhas é inconclusiva.
\item De acordo com o critério das linhas, o método de Gauss-Jacobi gera uma sequência convergente.


Partindo da solução inicial $x^{(0)}=\begin{bmatrix}
0&0&0
\end{bmatrix}^t$, as candidatas a solução são $x^{(1)}=\begin{bmatrix}
1.4&1&1.5
\end{bmatrix}^t$ e $x^{(2)}=\begin{bmatrix}
0.9&1.033&1
\end{bmatrix}^t$.

$d^{(1)}=1.5, d^{(2)}=0.5, d_r^{(1)}=1,d_r^{(2)}=0.484$.

$||Ax^{(1)}-b||_\infty=2.5, ||Ax^{(2)}-b||_\infty=0.467$.
\end{enumerate}

%\begin{enumerate}
%\item $\begin{bmatrix}
%12&9\\18&45
%\end{bmatrix}$
%\item $\begin{bmatrix}
%14&7\\32&43
%\end{bmatrix}$
%\item $\begin{bmatrix}
%14\\5\\4
%\end{bmatrix}$
%\item Não é possível calcular.
%\end{enumerate}
\end{Solucao}
\begin{Solucao}{2.2.2}
\begin{enumerate}
\item A análise pelo critério de Sasenfeld é inconclusiva.
\item De acordo com o critério de Sassenfeld, o método de Gauss-Seidel gera uma sequência convergente.

Partindo da aproximação inicial $x^{(0)}=\begin{bmatrix}
0&0&0
\end{bmatrix}^t$, temos $x^{(1)}=\begin{bmatrix}
1.4& 0.5333 &1.2333
\end{bmatrix}^t$ e $x^{(2)}=\begin{bmatrix}
1.0467& 1.0622 &0.9689
\end{bmatrix}^t$.
\end{enumerate}
\end{Solucao}
\begin{Solucao}{3.1.1}
\begin{enumerate}
\item A escolha do intervalo que possui raiz é livre. Para esta solução, irei supor que o intervalo escolhido foi $[-2,0]$. Pelo algoritmo da bissecção, as duas primeiras iterações geram os intervalos $[-1,0]$ e $[-1,-0.5]$. Com o intervalo escolhido, a candidata inicial $x_0$ é $-1$. As duas primeiras candidatas a solução calculadas são $x_1=-0.5$ e $x_2=-0.75$. A diferença absoluta entre essas candidatas são $d_1=0.5$ e $d_2=0.25$.
\end{enumerate}
\end{Solucao}

%\end{multicols}
\end{document}