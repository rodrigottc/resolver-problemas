%\documentclass[a4paper, 11pt, addpoints,answers]{exam}
%%\documentclass[a4paper, 11pt, addpoints, answers]{exam}  % Desconmente esta linha, para ver as respostas, e comente a de cima
%\usepackage{listaufersa}
%\usepackage{listings} % Mostrar código-fonte
%\usepackage[brazil]{babel}
%\usepackage{multicol}
%\usepackage{booktabs}
%\usepackage{hyperref}
%
%
%\usepackage{color}
%
%\setlength{\columnsep}{25pt}
%\lstdefinestyle{js}{
%    basicstyle=\ttfamily,
%    breaklines=true,
%    breakatwhitespace=true,
%    tabsize=1,
%    resetmargins=true,
%    xleftmargin=0pt,
%    frame=none
%}
%
%
%
%
%%%%%%%%%%%%%%%%%%%%%%%%%%%%%%%%%%%%%%%%%%%%%%%%%%
%
%\pointpoints{Ponto}{Pontos}

%\begin{document}
%
%
%%%%%%%%%% Informações sobre o curso %%%%%%%%%%%%%
%
%
%\nomeProfessor{Rodrigo Toledo Teixeira Câmara}
%\nomeCurso{Bacharelado em Ciências e Tecnologias}
%\nomeDisciplina{Cálculo Numérico}
%\semestre{5}   % Deixe em branco se for para mais de um semestre (recuperação)
%\dataDaProva{2016.2}
%\tipoAvaliacao{lista da aula 04}
%
%\info
%\vspace{-0.5 cm}
%
%
%%%%%%%%%%%%%%%%%%%%%%%%%%%%%%%%%%%%%%%%%%%%%%%%%



%\begin{questions}
%\begin{multicols}{2}


\begin{ex}
Considere o sistema
$$\begin{cases}
2x-y+3z+8w=-12\\
5y+z+w=17\\
4z-2w=10\\
10w=30
\end{cases}.$$
A matriz do coeficientes pode ser escrita como matriz triangular? Caso possa, encontre a solução. Caso contrário, aplique o método de Gauss para triangularizar este sistema.
\begin{sol}
Pode sim ser escrita como sistema triangular. No caso,
$$\begin{bmatrix}
1&-1&3&8\\
0&5&1&1\\
0&0&4&-2\\
0&0&0&10
\end{bmatrix}.$$
A solução é $X=\begin{bmatrix}
-46&2&4&3\end{bmatrix}^t$. 
\end{sol}
\end{ex}


\begin{ex}\label{sistema}
Considere o seguinte sistema linear:
$$\begin{bmatrix}
3	&2	&3\\
-1	&4	&6\\
10	&2	&12
\end{bmatrix}
\begin{bmatrix}
x_1\\x_2\\x_3
\end{bmatrix}=
\begin{bmatrix}
9\\32\\-2
\end{bmatrix}$$
\begin{enumerate}[label=\alph*)]
\item Resolva este sistema linear pelo método de eliminação de Gauss.\label{gauss}
\item Resolva este sistema pelo método da decomposição LU.
\item Resolva este sistema linear pelo método de eliminação de Gauss por meio do \emph{software Visual Cálculo Numérico}\footnote{Disponível gratuitamente no SIGAA.}. Configure o {\tt Formato dos Números} (disponível no menu {\tt Utilitários}) para a Notação Científica (Padrão em latino), com {\tt Precisão do número em casas decimais} ajustado para ``4''.
\end{enumerate}
\begin{sol}
\begin{enumerate}[label=\alph*)]
\item Primeiro reescrevo o sistema linear na forma da matriz aumentada,
$$\begin{bmatrix}
{\color{red}3}	& 2 & 3 & 9 \\
-1			& 4 & 6	& 32 \\
10			& 2 & 12 & -2 
\end{bmatrix}.$$
O pivô é o elemento $a_{11}$, pintado de vermelho. Vou usar este pivô para ``zerar'' todos os elementos abaixo dele, por meio das \emph{operações fundamentais}.

O primeiro passo é realizar as operações fundamentais $$L_2\leftarrow L_2 - \frac{-1}{{\color{red}3}}L_1$$ e $$L_3\leftarrow L_3 - \frac{10}{{\color{red}3}}L_1.$$ A matriz fica
$$\begin{bmatrix}
3			& 2 & 3 & 9 \\
0			& {\color{red}14/3} & 7	& 35 \\
0			& -14/3 & 2 & -32 
\end{bmatrix}.$$


O segundo passo é usar o pivô $14/3$ para zerar todos os elementos abaixo dele. A operação fundamental é $$L_3\leftarrow L_3 - \frac{-14/3}{{\color{red}14/3}}L_1.$$
A matriz fica
$$\begin{bmatrix}
3			& 2 & 3 & 9 \\
0			& 14/3 & 7	& 35 \\
0			& 0 & 9 & 3 
\end{bmatrix},$$
que é uma matriz triangular, fácil de resolver.

\item A matriz $A$, dada por $$A=\begin{bmatrix}
3	&2	&3\\
-1	&4	&6\\
10	&2	&12
\end{bmatrix},$$
pode ser decomposta de maneira única como produto de uma matriz inferior $L$ e uma matriz superior $U$ desta forma:
$$\begin{bmatrix}
3	&2	&3\\
-1	&4	&6\\
10	&2	&12
\end{bmatrix}=
\begin{bmatrix}
{\textcolor{red}1}	&{\textcolor{red}0}	&{\textcolor{red}0}\\
{\textcolor{blue}-1/3 }	&{\textcolor{red}1}	&{\textcolor{red}0}\\
{\textcolor{blue} 10/3 }	&{\textcolor{blue}-1}	&{\textcolor{red}1}
\end{bmatrix}\times 
\begin{bmatrix}
3	&2	&3\\
0	&14/3	&7\\
0	&0	&9
\end{bmatrix}.$$
Para construir a matriz $L$, eu tomei os {\textcolor{blue} multiplicadores} que foram utilizados no algoritmo de eliminação de Gauss. A matriz $U$ é simplesmente o resultado do algoritmo de Gauss.


O problema original pode ser reescrito como 
$$\begin{bmatrix}
{\textcolor{red}1}	&{\textcolor{red}0}	&{\textcolor{red}0}\\
{\textcolor{blue}-1/3 }	&{\textcolor{red}1}	&{\textcolor{red}0}\\
{\textcolor{blue} 10/3 }	&{\textcolor{blue}-1}	&{\textcolor{red}1}
\end{bmatrix}\times 
\begin{bmatrix}
3	&2	&3\\
0	&14/3	&7\\
0	&0	&9
\end{bmatrix}\times
\begin{bmatrix}
x_1\\x_2\\x_3
\end{bmatrix}=
\begin{bmatrix}
9\\32\\-2
\end{bmatrix}.$$
Agora precisamos apenas resolver os dois sistemas lineares \emph{triangulares}: o sistema $Ly=b$,
$$\begin{bmatrix}
{\textcolor{red}1}	&{\textcolor{red}0}	&{\textcolor{red}0}\\
{\textcolor{blue}-1/3 }	&{\textcolor{red}1}	&{\textcolor{red}0}\\
{\textcolor{blue} 10/3 }	&{\textcolor{blue}-1}	&{\textcolor{red}1}
\end{bmatrix}\times 
\begin{bmatrix}
a\\b\\c
\end{bmatrix}=
\begin{bmatrix}
9\\32\\-2
\end{bmatrix},$$
cuja solução é $a=9, b=35, c=3$, e este outro, $Ux=y$,
$$\begin{bmatrix}
3	&2	&3\\
0	&14/3	&7\\
0	&0	&9
\end{bmatrix}\times
\begin{bmatrix}
x_1\\x_2\\x_3
\end{bmatrix}=
\begin{bmatrix}
9\\35\\3
\end{bmatrix}
$$
cuja solução (que é a solução que procuramos) é $x_1=-2, x_2=7, x_3=1/3$.
\end{enumerate}
\end{sol}




\end{ex}

\begin{ex}
Aplique, se for possível, a fatoração LU na matriz
$$\begin{bmatrix}
2&3\\
4&5
\end{bmatrix}.$$
Utilize esta fatoração para resolver estes dois sistemas lineares:
$$\begin{cases}
2x+3y=16\\
4x+5y=26
\end{cases}\textnormal{ e }
\begin{cases}
2x+3y=5\\
4x+5y=9
\end{cases}.$$
\begin{sol}
A matriz $A$ dos coeficientes, dada por $\begin{bmatrix}
2&3\\4&5
\end{bmatrix}$, pode ser decomposta como
$$\begin{bmatrix}
2&3\\4&5
\end{bmatrix}=\begin{bmatrix}
1&0\\2&1
\end{bmatrix}\begin{bmatrix}
2&3\\0&-1
\end{bmatrix}.$$

A solução do primeiro sistema é $X=\begin{bmatrix}
1&1
\end{bmatrix}^t$. A solução do segundo sistema é $X=\begin{bmatrix}
-1&6
\end{bmatrix}^t$.
\end{sol}
\end{ex}

\begin{ex}
Considere que, ao final da primeira iteração do método de eliminação de Gauss, tenhamos o seguinte sistema linear:
$$\begin{bmatrix}
2	&8	&7	&7	\\
0	&1	&6	&1	\\
0	&3	&3	&1	\\
0	&-4	&1	&12	
\end{bmatrix}
\begin{bmatrix}
x_1\\
x_2\\
x_3\\
x_4
\end{bmatrix}=
\begin{bmatrix}
24\\8\\7\\9
\end{bmatrix}.$$
Se estivermos utilizando a estratégia do pivoteamento parcial, qual elemento escolheremos como pivô da segunda iteração?
\begin{sol}
$-4$.
\end{sol}
\end{ex}


\begin{ex}
Considere o seguinte sistema linear:\label{sistemainstavel}
$$\begin{bmatrix}
1\times 10^{-20}	&1\\
1					&-1\\
\end{bmatrix}
\begin{bmatrix}
x\\y
\end{bmatrix}=
\begin{bmatrix}
1\\0
\end{bmatrix}$$
\begin{enumerate}
\item Calcule a solução deste sistema em uma folha de papel. Simplifique as frações somente no final, quando for estipular a solução. \label{quebra}
\item Calcule a solução desse sistema novamente, mas dessa vez utilizando o \emph{software Visual Cálculo Numérico}. Configure o {\tt Formato dos Números} para a Notação Científica (Padrão em latino), com {\tt Precisão do número em casas decimais} configurado para ``4''. O resultado foi o mesmo encontrado no item \ref{quebra}? 
\end{enumerate}
\end{ex}

\begin{ex}
Resolva novamente o sistema mostrado na questão \ref{sistemainstavel}, com o \emph{Visual Cálculo Numérico} utilizando o pivoteamento parcial. Comente o resultado.
\end{ex}


\begin{ex}
Resolva os sistemas lineares abaixo pelo \emph{Visual Cálculo Numérico}, com 2 dígitos de precisão, com pivoteamento parcial e depois sem pivoteamento parcial. Compare as soluções.
\begin{enumerate}

\item $$\begin{cases}
0.21x+0.33y=0.54\\
0.70x+0.24y=0.94
\end{cases}.$$
\item $$\begin{cases}
0.11x-0.13y+0.20z=-0.02\\
0.10x+0.36y+0.45z=0.25\\
0.50x-0.01y+0.30z=-0.70
\end{cases}.$$
\end{enumerate}

\end{ex}




