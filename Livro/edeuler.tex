
\begin{ex}
Classifique as seguintes equações diferenciais em ``equações diferenciais ordinárias'' ou ``equações diferenciais parciais''. Em seguida, determine a ordem de cada uma.
%\begin{multicolumn}{3}
\begin{enumerate}
\item $y'=2x$
\item $y''+x^2(y')^3-4y=0$
\item $y'''+x^2+y^4=x \tan(x)$
\item $(1+xe^{xy})\frac{dy}{dx} + 1 + y e^{xy}=0$
\item $\frac{\partial f}{\partial x}+\frac{\partial f}{\partial y}=f^3(x,y)$
\item $\frac{\partial^2 u}{\partial t^2}+\frac{\partial u}{\partial t}=u(t)$
\end{enumerate}
%\end{multicolumn}
\end{ex}

\begin{ex}\label{geral}
Verifique que, para todo $c\in\mathbb{R}$, a função $y$ dada por $y(x)=ce^x+ x-1$ é solução da equação diferencial
$$y'=y-x+2.$$
Desenhe no \emph{Geogebra} alguns gráficos da função $y$ para alguns valores de $c$ (por exemplo, faça $c=1$, $c=2$, \emph{etc}. Se você souber usar o ``controle deslizante'' no \emph{Geogebra}, defina a variável $c$ desta forma).
\end{ex}


\begin{ex}\label{part}
Determine uma solução particular para a equação diferencial mostrada na questão \ref{geral}, tendo como condição inicial a equação $y(0)=2$. Faça o gráfico desta solução particular.
\end{ex}



\begin{ex}\label{pvi1}
Usando o método de Euler de ordem 1 e de ordem 2, determine a solução aproximada do PVI dado por
$$\begin{cases}
y'=f(x,y)=\frac{1}{x^2}-\frac{y}{x} - y^2\\
y(x_0)=y(1)=-1.
\end{cases},$$
onde o domínio de $y$ é $[1,2]$.
\begin{enumerate}
\item Considerando $h_1=\frac{1}{5}$ e $h_2=\frac{1}{10}$.
\item Sabendo-se que a solução analítica é dada por $y(x)=-\frac{1}{x}$ (verifique!), construa uma tabela que relaciona os valores exatos e os valores calculados pelos métodos de Euler de ordem 1 e 2.\label{tabela}
\end{enumerate}
\end{ex}


\begin{ex}
Seja $y$ uma função cujo domínio é $[0,2]$. Considere o seguinte PVI
$$\begin{cases}
y'=f(x,y)=y-x\\
y(x_0)=y(0)=2
\end{cases}.$$

\begin{enumerate}
\item Utilize os métodos de Euler de ordem 1 e de ordem 2 para achar uma aproximação da função $y$. Divida o domínio de $y$ em 2 partes e, em seguida, em 4 partes.
\item Construa uma tabela similar à construída no item \ref{tabela} da questão \ref{pvi1}.
\end{enumerate}
\end{ex}



\begin{ex}
Verifique se $y(x)=\frac{x^2+3}{3}$ é solução para o PVI
$$
\begin{cases}
y'=xy^{1/3}\\
y(0)=1
\end{cases}.
$$
\end{ex}

