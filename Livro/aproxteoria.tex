%\documentclass[a4paper, 11pt, addpoints]{exam}
%%\documentclass[a4paper, 11pt, addpoints, answers]{exam}  % Desconmente esta linha, para ver as respostas, e comente a de cima
%\usepackage{listaufersa}
%\usepackage{listings} % Mostrar código-fonte
%\usepackage[brazil]{babel}
%\usepackage{multicol}
%\usepackage{booktabs}
%\usepackage{hyperref}
%
%\setlength{\columnsep}{25pt}
%\lstdefinestyle{js}{
%    basicstyle=\ttfamily,
%    breaklines=true,
%    breakatwhitespace=true,
%    tabsize=1,
%    resetmargins=true,
%    xleftmargin=0pt,
%    frame=none
%}
%
%
%
%
%%%%%%%%%%%%%%%%%%%%%%%%%%%%%%%%%%%%%%%%%%%%%%%%%%
%
%\pointpoints{Ponto}{Pontos}
%
%
%
%
%
%\begin{document}
%
%

%%%%%%%%% Informações sobre o curso %%%%%%%%%%%%%

%
%\nomeProfessor{Rodrigo Toledo Teixeira Câmara}
%\nomeCurso{Bacharelado em Ciências e Tecnologias}
%\nomeDisciplina{Cálculo Numérico}
%\semestre{5}   % Deixe em branco se for para mais de um semestre (recuperação)
%\dataDaProva{2016.2}
%\tipoAvaliacao{lista da aula 08}
%
%\info
%\vspace{-0.5 cm}
%
%
%%%%%%%%%%%%%%%%%%%%%%%%%%%%%%%%%%%%%%%%%%%%%%%%%%
%
%
%
%\begin{questions}
%%\begin{multicols}{2}

\begin{ex}
Liste as diferenças entre aproximar uma função por \emph{interpolação polinomial} e aproximar função por \emph{ajuste pelo critério dos mínimos quadrados}.
\end{ex}




\begin{ex} 
Indique se a afirmação é verdadeira ou falsa.


\begin{enumerate}

\item\label{vf} \rule{1cm}{0.1pt} Ao se fazer a interpolação polinomial sobre $n$ pontos, o polinômio encontrado passa por todos este $n$ pontos.
\item \rule{1cm}{0.1pt} Ao se tabelar $n$ pontos de uma função podemos gerar polinômio de grau até $n$.
\item \rule{1cm}{0.1pt} Existem pelo menos três polinômios diferenes que passam pelos pontos tabelados: o polinômio obtido por resolução de sistema linear, o polinômio de Lagrange e o polinômio de Newton.
\item \rule{1cm}{0.1pt} Interpolação polinomial é uma aproximação de uma função.
\item \rule{1cm}{0.1pt} Quanto mais pontos são utilizados na interpolação melhor a aproximação será.
\item\label{vff} \rule{1cm}{0.1pt} Quanto mais pontos são utilizados no ajuste por curva melhor a aproximação será. 
\end{enumerate}
\end{ex}

%\begin{ex}
%Na questão 9 na lista da aula 4 foi pedido para se determinar, por meio de resolução de um sistema linear, qual função polinomial de segundo grau passa pelos pontos $(1;5)$, $(3;11)$ e $(5;25)$. 
%
%Utilize a técnica de Lagrange para determinar novamente esta função polinomial de segundo grau. Compare a dificuldade entre estas duas abordagens.
%\end{ex}

%\end{questions}
%
%\end{document}