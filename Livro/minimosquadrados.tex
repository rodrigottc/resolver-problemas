%\documentclass[a4paper, 11pt, addpoints]{exam}
%\documentclass[a4paper, 11pt, addpoints, answers]{exam}  % Desconmente esta linha, para ver as respostas, e comente a de cima
%\usepackage{listaufersa}
%\usepackage{listings} % Mostrar código-fonte
%\usepackage[brazil]{babel}
%\usepackage{multicol}
%\usepackage{booktabs}
%\usepackage{hyperref}
%
%\setlength{\columnsep}{25pt}
%\lstdefinestyle{js}{
%    basicstyle=\ttfamily,
%    breaklines=true,
%    breakatwhitespace=true,
%    tabsize=1,
%    resetmargins=true,
%    xleftmargin=0pt,
%    frame=none
%}
%
%
%
%
%%%%%%%%%%%%%%%%%%%%%%%%%%%%%%%%%%%%%%%%%%%%%%%%%%
%
%\pointpoints{Ponto}{Pontos}
%
%
%
%
%
%\begin{document}
%
%
%
%%%%%%%%%% Informações sobre o curso %%%%%%%%%%%%%
%
%
%\nomeProfessor{Rodrigo Toledo Teixeira Câmara}
%\nomeCurso{Bacharelado em Ciências e Tecnologias}
%\nomeDisciplina{Cálculo Numérico}
%\semestre{5}   % Deixe em branco se for para mais de um semestre (recuperação)
%\dataDaProva{2016.2}
%\tipoAvaliacao{lista da aula 11}
%
%\info
%\vspace{-0.5 cm}
%
%
%%%%%%%%%%%%%%%%%%%%%%%%%%%%%%%%%%%%%%%%%%%%%%%%%%
%
%
%
%\begin{questions}
%\begin{multicols}{2}

\begin{ex}
Dada a tabela \ref{experimentos.residuo} de pontos experimentais, obtenha a reta que melhor ajusta os pontos através do método dos mínimos quadrados. Em seguida, calcule o resíduo quadrado.
\begin{table}[hbt]
\centering
\caption{Valores obtidos em uma dada experiência.}
\label{experimentos.residuo}
\begin{tabular}{@{}llllll@{}}
\toprule
$x$	& 1	& 2   & 3		& 4   & 5          \\ \midrule
$f(x)$		& 2.2	& 3.3 & 4.2	& 5.1 & 6.3   \\ \bottomrule
\end{tabular}
\end{table}

\end{ex}

%\begin{ex}
%A tabela \ref{sen} traz alguns valores da função trigonométrica \emph{seno}:
%\begin{table}[hbt]
%\centering
%\caption{Cálculo de alguns valores da função \emph{seno}.}
%\label{sen}
%\begin{tabular}{@{}llllll@{}}
%\toprule
%$x$ em radianos	& 0	& 0.05   & 0.1		& 0.15   & 0.2          \\ \midrule
%$\sin(x)$		& 0	& 0.0500 & 0.0998	& 0.1494 & 0.1987   \\ \bottomrule
%\end{tabular}
%\end{table}
%
%
%Ajuste estes pontos a uma reta e a uma parábola utilizando o método dos mínimos quadrados. Qual se ajustou melhor?
%\end{ex}

\begin{ex}
Considere a tabela \ref{cidade} de observações do consumo de água em uma cidade de 20 000 habitantes.

% Please add the following required packages to your document preamble:
% \usepackage{booktabs}
\begin{table}[htb]
\centering
\caption{Consumo de água em uma cidade ao longo do tempo}
\label{cidade}
\begin{tabular}{@{}llllll@{}}
\toprule
Dia     & 1    & 2    & 3    & 4    & 5    \\ \midrule
Consumo & 11.0 & 10.5 & 10.3 & 10.2 & 10.2 \\ \bottomrule
\end{tabular}
\end{table}
Considere duas estratégias para prever o consumo de água:
\begin{enumerate}
\item Interpole a função nos dois últimos períodos, 3 e 4 (obtenha, portanto, uma reta) e use este polinômio interpolador para prever o consumo no dia 5. \label{poliagua}
\item Ajuste uma reta, pelo método dos mínimos quadrados, aos três últimos períodos (dias 2, 3 e 4) e use esta reta para prever o consumo no dia 5. \label{quadagua} 
\end{enumerate}
Observe as previsões para o dia 5 que você obteve nos itens \ref{poliagua} e \ref{quadagua} e compare com o valor ``real'' do consumo do dia 5, mostrado na tabela. Qual das duas estratégias apresentou uma previsão melhor?
\end{ex}

\begin{ex}
Construa uma tabela com os valores de $e^x$ para $x_k=\frac{k}{5}$, com $k\in\{1,...,5\}$ e determine o valor aproximado de $e^{0.23}$ usando
\begin{enumerate}
\item Polinômio interpolador,
\item Ajuste de curvas, utilizando uma parábola.
\end{enumerate} 
\end{ex}
\subsubsection{Problemas}
\begin{ex}
Um grupo de alunos de biologia estudou a bioquímica de uma certa espécie de alga, a \emph{Chlorophyta Caulerpa cupressoides}. Eles observaram que a salinidade (em $PSU$) da água influencia nas atividades biológicas da alga, em especial na síntese de uma substância chamada \emph{polissacarimídia} (em milimol por hora). Os resultados das experiências estão descritos na tabela \ref{alga}.

% Please add the following required packages to your document preamble:
% \usepackage{booktabs}
\begin{table}[htb]
\centering
\caption{Experiências que relacionam a capacidade da alga \emph{Chlorophyta Caulerpa cupressoides} sintetizar polissacarimídia em função da salinidade.}
\label{alga}
\begin{tabular}{@{}cc@{}}
\toprule
Salinidade & Síntese de polissacarimídia \\ \midrule
3.2        & 0.9                         \\
0.6        & 3.9                         \\
1.3        & 2.8                         \\
2.3        & 2.1                         \\
3.1        & 1.6                         \\ \bottomrule
\end{tabular}
\end{table}
Ao ajustar estes dados por uma curva hiperbólica, obtém-se $\varphi(x)=1/(0.3x+0.1)$. Ao se ajustar por uma reta, encontra-se $\varphi(x)=-x+4.3$. Segundo o critério dos mínimos dos desvios quadrados, qual dessas duas curvas se ajusta melhor a este fenômeno biológico? 
\end{ex}



%\end{questions}
%
%\end{document}