
\begin{question}
Considere a função dada $f$ dada por $$f(x)=x^6+4x^2-20.$$
\begin{parts}
\part Trace o gráfico desta função no \emph{GeoGebra}  e estime a menor raiz positiva. 
\part Utilizando a forma geral $$\varphi(x)=x+A(x)f(x),$$ com $A(\xi)\neq 0$, construa três funções iterativas $\varphi$.
\part Verifique se alguma destas funções iterativas converge para a menor raiz positiva de $f$.
\part Determine a função $\varphi$ conforme proposta pelo método de Newton-Raphson.
\part Calcule as duas primeiras iterações deste método.
\end{parts}
\end{question}

\begin{question}\label{newton}
Considere a função $f$ dada por
$$ f(x) = x^8 + 6x^2 - 4x + 0.6655.$$ 
\begin{parts}
\part Trace o gráfico desta função no \emph{GeoGebra}  e estime a menor raiz positiva.
\part Determine, utilizando o método de Newton-Raphson e precisão $\epsilon=0.01$, a raiz $\overline{x}$ desta função com o seguinte critério de parada:
\begin{subparts}
\subpart Diferença absoluta entre duas candidatas a solução.
\subpart Diferença absoluta entre $f(\overline{x})$ e $0$.
\end{subparts}
Comente cada um dos resultados.
\end{parts}
\end{question}

\begin{question}
Calcule as duas primeiras iterações do método da secante das funções dadas abaixo. Esboce os gráficos das funções no \emph{GeoGebra} para estimar as duas aproximações iniciais da menor raiz positiva.
\begin{parts}
\part $f(x)=x^2-10.$
\part $g(x)=x\sin(x).$
\part $h(x)=(x-2)^2-\ln(x).$
\end{parts}
\end{question}

\begin{question}
Encontre $x$ tal que $e^x=x^2$.
\end{question}

\begin{question}
Encontre a raiz quadrada de 7 com treze algarismos significativos. Dica: a raiz quadrada de $a$ é encontrada ao se calcular a raiz da função $f(x)=x^2-a$.
\end{question}

\begin{question}
Encontre a raiz cúbica de 10 com treze algarismos significativos. Dica: a raiz $n$-ésima de $a$ é encontrada ao se calcular a raiz da função $f(x)=x^n-a$.
\end{question}

%\begin{question}
%A velocidade de um paraquedista, após abrir o paraquedas, é dada pela equação
%$$v(t)=\frac{gm(1-e^{-(\frac{ct}{m})})}{c},$$
%onde \begin{itemize}
%\item $v$ é a velocidade absoluta em $m/s$ do paraquedista, 
%\item $g$ é a aceleração da gravidade, 
%\item $m$ é a massa do paraquedista (já com o equipamento), 
%\item $t$ é o tempo em segundos após a abertura do paraquedas e 
%\item $c$ é o coeficiente de amortecimento, em $kg/s$, que depende, principalmente, do paraquedas.
%\end{itemize}
%Considerando a aceleração da gravidade como $9.8 m/s^2$, a massa do paraquedista com o equipamento é 100$kg$ e que ele atinge a velocidade de $40 m/s$ sete segundos após abrir o paraquedas, encontre o coeficiente de amortecimento deste paraquedas.
%\end{question}

\begin{question}
A equação que relaciona a taxa de juros $i$ com o valor à vista $PV$, o valor fixo\footnote{Este método de pagamento, que envolve prestações iguais, é conhecido como o \emph{sistema Price}.} $PMT$ de cada parcela e a quantidade de parcelas $n$ é
$$PMT=PV\frac{i(1+i)^n}{(1+i)^n-1}.$$
Supondo que uma geladeira é vendida à vista por R$\$1500,00$ e que a loja permite o pagamento em doze parcelas iguais de R$\$150,00$, calcule a taxa de juros deste financiamento. 
\end{question}

\begin{question}
Um espectrômetro analisou as ondas de uma certa transmissão de rádio e as descreveu como a seguinte função senoidal:
$$y(x)=x^3\sin(x)+\cos(x)+3.$$
Trace esta função no \emph{GeoGebra} para observar uma representação visual destas ondas da rádio. 


O \emph{comprimento de onda}, $\lambda$, em metros, é definido, neste contexto, como o dobro da distância horizontal entre duas raízes consecutivas desta função. A \emph{frequência} $f$, em hertz, é dada por
$$f=\frac{c}{\lambda},$$
onde $c$ é a velocidade da luz, dada por $c=3\times 10^8 m/s$.


Determine a frequência desta transmissão de rádio.
\end{question}
