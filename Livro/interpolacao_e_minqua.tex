%\documentclass[a4paper, 11pt, addpoints]{exam}
%%\documentclass[a4paper, 11pt, addpoints, answers]{exam}  % Desconmente esta linha, para ver as respostas, e comente a de cima
%\usepackage{listaufersa}
%\usepackage{listings} % Mostrar código-fonte
%\usepackage[brazil]{babel}
%\usepackage{multicol}
%\usepackage{booktabs}
%\usepackage{hyperref}
%
%\setlength{\columnsep}{25pt}
%\lstdefinestyle{js}{
%    basicstyle=\ttfamily,
%    breaklines=true,
%    breakatwhitespace=true,
%    tabsize=1,
%    resetmargins=true,
%    xleftmargin=0pt,
%    frame=none
%}
%
%
%
%
%%%%%%%%%%%%%%%%%%%%%%%%%%%%%%%%%%%%%%%%%%%%%%%%%%
%
%\pointpoints{Ponto}{Pontos}
%
%
%
%
%
%\begin{document}
%
%
%
%%%%%%%%%% Informações sobre o curso %%%%%%%%%%%%%
%
%
%\nomeProfessor{Rodrigo Toledo Teixeira Câmara}
%\nomeCurso{Bacharelado em Ciências e Tecnologias}
%\nomeDisciplina{Cálculo Numérico}
%\semestre{5}   % Deixe em branco se for para mais de um semestre (recuperação)
%\dataDaProva{2016.2}
%\tipoAvaliacao{lista da aula 11}
%
%\info
%\vspace{-0.5 cm}
%
%
%%%%%%%%%%%%%%%%%%%%%%%%%%%%%%%%%%%%%%%%%%%%%%%%%%
%
%
%
%\begin{questions}
%\begin{multicols}{2}

\begin{ex}
Dada a tabela \ref{experimentos} de pontos experimentais, obtenha a reta que melhor ajusta os pontos através do método dos mínimos quadrados. Em seguida, calcule o resíduo quadrado.
\begin{table}[hbt]
\centering
\caption{Valores obtidos em uma dada experiência.}
\label{experimentos}
\begin{tabular}{@{}llllll@{}}
\toprule
$x$	& 1	& 2   & 3		& 4   & 5          \\ \midrule
$f(x)$		& 2.2	& 3.3 & 4.2	& 5.1 & 6.3   \\ \bottomrule
\end{tabular}
\end{table}

\end{ex}

\begin{ex}
A tabela \ref{sen} traz alguns valores da função trigonométrica \emph{seno}:
\begin{table}[hbt]
\centering
\caption{Cálculo de alguns valores da função \emph{seno}.}
\label{sen}
\begin{tabular}{@{}llllll@{}}
\toprule
$x$ em radianos	& 0	& 0.05   & 0.1		& 0.15   & 0.2          \\ \midrule
$\sin(x)$		& 0	& 0.0500 & 0.0998	& 0.1494 & 0.1987   \\ \bottomrule
\end{tabular}
\end{table}


Ajuste estes pontos a uma reta e a uma parábola utilizando o método dos mínimos quadrados. Qual se ajustou melhor?
\end{ex}

%
\begin{ex}
Construa uma tabela com os valores de $e^x$ para $x_k=\frac{k}{5}$, com $k\in\{1,...,5\}$ e determine o valor aproximado de $e^{0.23}$ usando
\begin{enumerate}
\item Polinômio interpolador,
\item Ajuste de curvas, utilizando uma parábola.
\end{enumerate} 
\end{ex}




%\end{questions}
%
%\end{document}